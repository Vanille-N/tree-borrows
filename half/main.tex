\documentclass[a4paper,11pt]{article}

\usepackage[a4paper,margin=1in]{geometry}
\usepackage[utf8]{inputenc}
\usepackage{amsmath}
\usepackage{amssymb}
\usepackage{amsthm}
\usepackage{booktabs}
\usepackage[small]{caption}
\usepackage{cite}
\usepackage{colortbl}
\usepackage{enumitem}
\usepackage{framed}
\usepackage{graphicx}
\usepackage{microtype}
\usepackage{xcolor}
\usepackage[unicode]{hyperref}
\usepackage{amsfonts}
\usepackage{algorithm}
\usepackage{algpseudocode}
\usepackage{bm}
\usepackage{mathtools}
\usepackage{subcaption}

\usepackage{listings} % replace with nice syntax hl

\bibliographystyle{plainurl}

\setlength{\OuterFrameSep}{0.3ex}
\setlength{\FrameSep}{1.5ex}

\newcommand{\myaff}[1]{\,$\cdot$\, {\small #1}\par\smallskip}
\newcommand{\fakeparagraph}[2]{\par\noindent\textbf{#1}\hspace{1em}#2}

\usepackage{thm-restate}
\usepackage{cite}

% Problem definition environment
\newfloat{problemdef}{htbp}{loa}
\floatname{problemdef}{Problem}
\newcommand{\problemcaptionkludge}{\rule[-.3\baselineskip]{0pt}{\baselineskip}}

\theoremstyle{plain}
\newtheorem{theorem}{Theorem}[section]
\newtheorem{lemma}[theorem]{Lemma}
\newtheorem{corollary}[theorem]{Corollary}
\newtheorem{observation}[theorem]{Observation}
\theoremstyle{definition}
\newtheorem{definition}[theorem]{Definition}
\newtheorem{example}[theorem]{Example}
\theoremstyle{remark}
\newtheorem*{remark}{Remark}
\newtheorem{openquestion}[theorem]{Open question}

% Well-parenthesized commands
\newcommand{\set}[1]{\ensuremath{\left\{#1\right\}}}

\newenvironment{myabstract}
{\list{}{\listparindent 1.5em%
        \itemindent    \listparindent
        \leftmargin    0cm
        \rightmargin   0cm
        \parsep        0pt}%
    \item\relax}
{\endlist}

\newenvironment{mycover}
{\list{}{\listparindent 0pt
        \itemindent    \listparindent
        \leftmargin    0cm
        \rightmargin   0cm
        \parsep        0pt}%
    \raggedright
    \item\relax}
{\endlist}

\begin{document}

\begin{mycover}
{\huge\bfseries\boldmath Tree Borrows\par}
\bigskip
\bigskip
\bigskip


Author: \textbf{Neven Villani}
\myaff{ENS Paris-Saclay}

~\newline

Advisor: \textbf{Derek Dreyer}
\myaff{MPI for Software Systems}


Advisor: \textbf{Ralf Jung}
\myaff{ETH Zürich}


\end{mycover}
\medskip

\begin{myabstract}
\fakeparagraph{Abstract.}
\end{myabstract}
\medskip


\section{Introduction}

While the purpose of type systems and typing information in programs is usually
presented as primarily a matter of security (strict type systems can rule out
at compilation-time a number of bugs), they also enable compilers to generate
more efficient code in both space and time. Languages with strict compile-time
type systems can avoid the need for typing metadata at runtime, have fewer
bounds checks for memory accesses, or even in the case of Rust eliminate the
need for a runtime garbage collector entirely.

In Rust the type system includes aliasing information (mutability and uniqueness),
which is to be used not only for safety guarantees, but also to improve
performance and enable optimizations that are only valid under certain aliasing
guarantees.
We aim to define formally what these aliasing guarantees are for Rust, as
well as show how to check them and use them in proving the validity of compiler
optimizations.

\subsection{Motivating example}

As a first concrete example, consider the following function:
\begin{lstlisting}
fn example1(x: &mut u64, y: &mut u64) -> u64 {
     let xval = *x; // First read of *x
     *y = xval + 1;
     let xval = *x; // Second (redundant ?) read of *x
     return xval;
}
\end{lstlisting}

Because mutable references in Rust are required to be unique, \texttt{x} and
\texttt{y} must point to disjoint regions of memory. In particular the
instruction \texttt{*y = xval + 1} constitutes a write to \texttt{y}, but it
cannot affect the memory covered by \texttt{x}: the value \texttt{*x} is unaffected
and thus the second read of \texttt{*x} is redundant. This function can be
optimized to perform one fewer operation by deleting the second line on which
\texttt{*x} is read, without modifying behavior.

Unfortunately there is a huge gap in the proof above: we have not specified
by whom, in what contexts or in which manner mutable references are ``required''
to be unique. This is the purpose of Tree Borrows.

Indeed there exists in Rust the \texttt{unsafe} keywords which
allows the programmer to bypass certain compiler checks by extending their
available instruction set: among other things the \texttt{unsafe} keyword allows
dereferencing raw pointers in the following manner:
\begin{lstlisting}
fn context1() {
    let mut data = 42u64;
    let data_ptr = &mut data as *mut u64;
    let x: &mut u64 = unsafe { &mut *data_ptr };
    let y: &mut u64 = unsafe { &mut *data_ptr };
    let result = example1(x, y);
    assert!(result == 43);
}
\end{lstlisting}
Here we use \texttt{unsafe} instructions to obtain two mutable references
to the same location, and we pass both of them to \texttt{example1}.
While in this context the original version of \texttt{example1} will return
\texttt{43}, the optimized version will instead return \texttt{42}.

The optimization shown here is thus not unconditionally valid: violating the
unicity requirement of mutable references has enabled us to create a program
in which the optimization does not preserve behavior.

However we want this optimization to be valid, on the grounds of \texttt{context1}
being a ``bad'' program that violates some assumptions that we want to be able
to make. This issue is solved by adjusting the operational semantics of Rust
in a way that declares \texttt{context1} ``Undefined Behavior'' (UB):
a program exhibits UB when it performs some forbidden instruction.
Compilers can assume that programs do not exhibit UB, and optimizations are
not required to preserve the observable behavior of such programs.

There is a tradeoff in what programs can be declared UB, indeed for compiler writers
the more programs are declared UB the more powerful optimizations can be made and
the more freedom there is in what is considered a correct compiler, while for language
users the more programs are declared UB the less the execution closely matches the source
code. Consider for example the two extreme cases:
\begin{itemize}
    \item if all programs are UB then it is valid to compile all programs as an
        empty sequence of instructions, optimizations are too powerful and the language
        is so underspecified as to become useless;
    \item on the other hand if no program is UB, then few to no optimizations are possible.
\end{itemize}
Both sides however have an interest in the rules governing UB being clear and well-defined:
if the rules are vague then compiler writers are unsure what assumptions they can make,
and language users may accidentally write a program that exhibits UB.

An explicit design decision of Rust is that programs that that do not use the
\texttt{unsafe} keyword cannot be declared UB. For \texttt{unsafe} to be useful,
it should also hold that it is not too difficult to write unsafe Rust that is not
UB. Tree Borrows should thus:
\begin{itemize}
    \item \textbf{Declare enough programs UB that some useful optimizations are valid.}
        We evaluate Tree Borrows on this aspect by providing proofs of validity of such
        optimizations using the aliasing assumptions allowed by the Tree Borrows semantics.
    \item \textbf{Declare as little UB as possible in programs that have already been written.}
        Each program retroactively declared to exhibit UB is a violation of backwards compatibility,
        we must ensure that these are few and justifiable.
        To check this aspect we implement the Tree Borrows semantics in the Miri
        interpreter and run the OS-independent part of the Rust standard library
        test suite using this semantics. We found very few rejected instances, and
        the suggested changes were accepted and merged back into the standard library.
    \item \textbf{Have rules that are simple, consistent and intuitive.}
        While it is difficult to measure this objectively, we argue that compared
        to its predecessor Stacked Borrows, Tree Borrows is more consistent in
        its handling of pointers. In particular a number of bug reports on the
        Github page of Miri show that users misunderstand some details of the
        behavior of Stacked Borrows, and we have ensured that such misunderstandings
        are not present in Tree Borrows.
    \item \textbf{Be efficient.} In addition to being humanly understandable, Tree Borrows
        should also be verifiable in practice without too much overhead in Miri.
        We compare it to benchmarks of Stacked Borrows and observe a noticeable slowdown,
        but not to the point that it would be an obstacle to the usability of Miri.
\end{itemize}

\section{Aliasing in Rust}

In this section we explain the features of Rust that we are interacting with --
mostly the different kinds of pointers available -- and provide an intuition
for the aliasing constraints they should satisfy.

We call ``safe Rust'' the subset of the complete Rust language (more explicitly called ``unsafe Rust'')
that does not use the \texttt{unsafe} keyword (and thus does not use any \texttt{unsafe}
instructions).

\subsection{References and raw pointers}

Rust offers a kind of pointer with more guarantees than its counterpart in most
languages: references. Written \texttt{\&} (\texttt{let x:~\&T = \&t;}) or
\texttt{\&mut} (\texttt{let x:~\&mut T = \&mut t;}) for immutable and mutable
references respectively.

These references should satisfy ``aliasing XOR mutability'': for a given memory
location, if a mutable reference exists there cannot also be other mutable or
immutable references. We can have access to either unique write permissions or
shared read permissions, not both.


As a means of interfacing with C code and expressing complex pointer manipulations,
Rust also offers another type of pointers, called raw pointers, written
\texttt{*const T} for immutable raw pointers and \texttt{*mut T} for mutable
raw pointers. There pointers can bypass some requirements of references (there
can exist multiple \texttt{*mut T} to the same location), but their use is more
dangerous and must be enclosed by the keyword \texttt{unsafe \{...\}} thus
their use is usually restricted to situations where they are absolutely necessary.

Although several \texttt{*mut T} can exist simultaneously, it is still forbidden
for them to coexist with \texttt{\&mut T}.


The following snippet shows some conversions between these kinds of pointers:
\begin{lstlisting}
let mut data = 42u64;
let some_mut_ref: &mut u64 = &mut data;
let const_from_mut: &u64 = &*some_mut_ref;
let reborrow_mut: &mut u64 = &mut *some_mut_ref;
let mut_ptr: *mut u64 = &mut *some_mut_ref as *mut u64;
let ref_from_ptr: &mut u64 = unsafe { &mut *mut_ptr };
\end{lstlisting}

\section{Blind spots and limitations of existing tools}

\subsection{The Borrow Checker}

The Borrow Checker is a compile-time verification of some aliasing rules, and
in the presence of safe Rust it is able to guarantee that mutable references
have exclusive access and that shared references have access to data that will
not be mutated. However it is in several aspects not fine-grained enough compared
to the model we want to develop.

\paragraph*{It does not check \texttt{unsafe} code.}

\paragraph*{It cannot use runtime properties.}

\subsection{Stacked Borrows}

\paragraph*{Accesses, not creations, are the actual violations.}

\paragraph*{Reads should not invalidate other Reads.}

\paragraph*{On handling pointee types of unknown size.}

\section{The Tree Borrows aliasing model}

\subsection{A tree of pointers}

\subsection{Permissions and updates}

\section{Tree Borrows implemented}

\subsection{Performance concerns and optimizations}

\subsection{Testing the Rust Standard Library}

\section{Future work}

\subsection{Formalization}

\subsection{Proving optimizations}


%%
%% Bibliography
%%

\bibliography{literature}


\newpage
\appendix

\end{document}
